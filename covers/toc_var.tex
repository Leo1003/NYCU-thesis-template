% 縮小目錄、標題上方的空白處
\titleformat{\chapter}[display]{\normalfont\huge\bfseries}{}{0pt}{\centering}
\titlespacing*{\chapter}{0pt}{-40pt}{40pt}


\iftoggle{toc-use-cn}
{ % true section. 使用中文

% 下面這些是要在目錄上加入...的符號與頁碼
%\titlecontents{chapter}[0em]{}{\thecontentslabel \hspace{1em}}{}{\titlerule*{.}\contentspage}[\addvspace{1em}]
%\titlecontents{section}[1.5em]{\addvspace{-0.5em}}{\thecontentslabel \hspace{1em}}{}{\titlerule*{.}\contentspage}[\addvspace{0.5em}]
%\titlecontents{subsection}[3em]{}{\thecontentslabel \hspace{1em}}{}{\titlerule*{.}\contentspage}[\addvspace{0.5em}]

\titlecontents{chapter}[0em]{}{第\CJKnumber{\thecontentslabel}章 \hspace{0.5em}}{}{\titlerule*{.}\contentspage}[\addvspace{1em}]
\titlecontents{section}[1.5em]{\addvspace{-0.5em}}{\thecontentslabel \hspace{1em}}{}{\titlerule*{.}\contentspage}[\addvspace{0.5em}]
\titlecontents{subsection}[3em]{}{\thecontentslabel \hspace{1em}}{}{\titlerule*{.}\contentspage}[\addvspace{0.5em}]

% 6. 致謝 Acknowledgement
\phantomsection
\addcontentsline{toc}{chapter}{誌謝}
% --- Acknowledgment ---
\begin{center}
\Large
\textbf{誌~~~~~~謝}
\end{center}

\vspace{1cm}
\linespread{2}%
\selectfont
\hspace{0.25cm}

% 下面開始寫致謝內容
謝天謝地

\vspace{3cm}
\begin{flushright}
XXXXX於

國立交通大學\NameofDepartmentInstituteCN

\ThesisDateTW
\end{flushright}
\newpage

% 7 中文摘要 chinese abstract
\phantomsection
\addcontentsline{toc}{chapter}{摘要}

  \begin{center}
	\large
    \begin{singlespace}    
      \textbf{\chineseTitle{}} \\[0.5cm]
    \end{singlespace}
    
    \begin{singlespace}    

    學生:\studentCnName  \hspace{2.5cm}  指導教授:\advisorCnName \hspace{0.1cm} 博士 \\
    \ifdefined\advisorCnNameB % 如果有共同指導教授
    \hspace{9.6cm}  \advisorCnNameB ~ 博士 \\
    \fi
    \end{singlespace}

    \vspace{0.5cm}

    國立陽明交通大學\ \DepartInstitCnName\ \iftoggle{iamphd}{博士班}{碩士班} \\[0.5cm]
    \textbf{摘~~~~~~~~要} \\[0.5cm]

  \end{center}
  \normalsize 
  %\hspace{0.75cm}
  中文摘要就從這邊開始寫.

  \vspace{1cm}

  % 中文摘要及關鍵詞 5-7 個 
  \textbf{關鍵字:}中文, 摘要, 關鍵詞, 5-7個, 不要多, 也不要少

\newpage

% 8. 英文摘要
\phantomsection
\addcontentsline{toc}{chapter}{Abstract}
\begin{center}
    \large
    
    \begin{singlespace}
        \textbf{\englishTitle{}} \\[0.5cm]
    \end{singlespace}
    
    \begin{singlespace}
        Student : \studentEnName{}  \hspace{1.0cm} 
        % 兩個指導教授要寫 Advisors
        \ifdefined\advisorCnNameB
            Advisors: Dr.\, \advisorEnName \\
            \hspace{6.6cm} Dr.\, \advisorEnNameB  \\
        \else
            Advisor: Dr.\, \advisorEnName \\
        \fi
    \end{singlespace}
    
    \vspace{0.5cm}
    \begin{singlespace}
        \NameofDepartmentInstituteEN\\
        National Yang Ming Chiao Tung University\\[0.2cm]
    \end{singlespace}
    
    \textbf{Abstract} \\[0.5cm]

\end{center}

\normalsize 
  
Write your abstract here. Through computer vision technologies, ...


\vspace{1cm}

% 5-7 Keywords (English) 
\textbf{Keywords: English, keywords, five to seven, computer vision, IoT.} 

\newpage

% 9. 目錄 中文版
\renewcommand{\contentsname}{目錄} % 使用中文目錄
\phantomsection
\addcontentsline{toc}{chapter}{目錄}
\tableofcontents
\newpage


% 10. 圖片目錄 中文版
\renewcommand{\figurename}{圖} % 把caption的Figure改成"圖"
\renewcommand{\listfigurename}{圖目錄}
\renewcommand{\numberline}[1]{圖~#1\hspace*{1em}}
\phantomsection
\addcontentsline{toc}{chapter}{圖目錄}
\listoffigures
\newpage

% 11. 表格目錄 中文版, 有需要再打開
\renewcommand{\tablename}{表} % 把caption的Table改成"表"
\renewcommand{\listtablename}{表目錄}
\renewcommand{\numberline}[1]{表~#1\hspace*{1em}}
\phantomsection
\addcontentsline{toc}{chapter}{表目錄}
\listoftables
\newpage

% 把Chapter改成 第X章
% 如果想要章節標題置中, 可在\normalfont前面加上\centering
\titleformat{\chapter}{\normalfont\huge\bfseries}{第\CJKnumber{\thechapter}章 \hspace{0.5em}}{0em}{}
\titleformat{\section}{\normalfont\Large\bfseries}{\thesection}{1em}{}
\titleformat{\subsection}{\normalfont\large\bfseries}{\thesubsection}{1em}{}

} % true end. 中文目錄設定結束
% --------------------------------------------
% --------------------------------------------
{ % false section. 使用英文目錄
% 下面這些是要在目錄上加入...的符號與頁碼
\titlecontents{chapter}[0em]{}{\thecontentslabel \hspace{1em}}{}{\titlerule*{.}\contentspage}[\addvspace{1em}]
\titlecontents{section}[1.5em]{\addvspace{-0.5em}}{\thecontentslabel \hspace{1em}}{}{\titlerule*{.}\contentspage}[\addvspace{0.5em}]
\titlecontents{subsection}[3em]{}{\thecontentslabel \hspace{1em}}{}{\titlerule*{.}\contentspage}[\addvspace{0.5em}]

% 6. 致謝 Acknowledgement
\addcontentsline{toc}{chapter}{Acknowledgement}
% --- Acknowledgment ---
\begin{center}
\Large
\textbf{誌~~~~~~謝}
\end{center}

\vspace{1cm}
\linespread{2}%
\selectfont
\hspace{0.25cm}

% 下面開始寫致謝內容
謝天謝地

\vspace{3cm}
\begin{flushright}
XXXXX於

國立交通大學\NameofDepartmentInstituteCN

\ThesisDateTW
\end{flushright}\newpage

% 7 中文摘要 chinese abstract
\addcontentsline{toc}{chapter}{Chinese Abstract} 
  \begin{center}
	\large
    \begin{singlespace}    
      \textbf{\chineseTitle{}} \\[0.5cm]
    \end{singlespace}
    
    \begin{singlespace}    

    學生:\studentCnName  \hspace{2.5cm}  指導教授:\advisorCnName \hspace{0.1cm} 博士 \\
    \ifdefined\advisorCnNameB % 如果有共同指導教授
    \hspace{9.6cm}  \advisorCnNameB ~ 博士 \\
    \fi
    \end{singlespace}

    \vspace{0.5cm}

    國立陽明交通大學\ \DepartInstitCnName\ \iftoggle{iamphd}{博士班}{碩士班} \\[0.5cm]
    \textbf{摘~~~~~~~~要} \\[0.5cm]

  \end{center}
  \normalsize 
  %\hspace{0.75cm}
  中文摘要就從這邊開始寫.

  \vspace{1cm}

  % 中文摘要及關鍵詞 5-7 個 
  \textbf{關鍵字:}中文, 摘要, 關鍵詞, 5-7個, 不要多, 也不要少
 \newpage

% 8. english abstract
\addcontentsline{toc}{chapter}{English Abstract} \begin{center}
    \large
    
    \begin{singlespace}
        \textbf{\englishTitle{}} \\[0.5cm]
    \end{singlespace}
    
    \begin{singlespace}
        Student : \studentEnName{}  \hspace{1.0cm} 
        % 兩個指導教授要寫 Advisors
        \ifdefined\advisorCnNameB
            Advisors: Dr.\, \advisorEnName \\
            \hspace{6.6cm} Dr.\, \advisorEnNameB  \\
        \else
            Advisor: Dr.\, \advisorEnName \\
        \fi
    \end{singlespace}
    
    \vspace{0.5cm}
    \begin{singlespace}
        \NameofDepartmentInstituteEN\\
        National Yang Ming Chiao Tung University\\[0.2cm]
    \end{singlespace}
    
    \textbf{Abstract} \\[0.5cm]

\end{center}

\normalsize 
  
Write your abstract here. Through computer vision technologies, ...


\vspace{1cm}

% 5-7 Keywords (English) 
\textbf{Keywords: English, keywords, five to seven, computer vision, IoT.} 
 \newpage

% 9. 目錄 English version
\renewcommand{\contentsname}{Contents}
\addcontentsline{toc}{chapter}{Contents} \tableofcontents \newpage

% 10. 圖片目錄 English version
\renewcommand{\listfigurename}{List of Figures}
\renewcommand{\numberline}[1]{Figure~#1\hspace*{1em}}
\addcontentsline{toc}{chapter}{List of Figures} \listoffigures \newpage

% 11. 表格目錄 English version, 有需要再打開
\renewcommand{\listtablename}{List of Tables}
\renewcommand{\numberline}[1]{Table~#1\hspace*{1em}}
\addcontentsline{toc}{chapter}{List of Tables} \listoftables \newpage

% 調整內文的chapter, section, subection的顯示方式, 改成靠左對齊+沒有換行
% 如果想要章節標題置中, 可在\normalfont前面加上\centering
\titleformat{\chapter}{\normalfont\huge\bfseries}{Chapter {\thechapter}.}{1em}{}
\titleformat{\section}{\normalfont\Large\bfseries}{\thesection}{1em}{}
\titleformat{\subsection}{\normalfont\large\bfseries}{\thesubsection}{1em}{}

} % false end. 使用英文
